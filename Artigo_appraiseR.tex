\documentclass[article]{jss}
\usepackage[utf8]{inputenc}

\providecommand{\tightlist}{%
  \setlength{\itemsep}{0pt}\setlength{\parskip}{0pt}}

\author{
Luiz Fernando Palin Droubi\\SPU \And Willian Zonato\\SPU \And Norberto Hoccheim\\UFSC
}
\title{\pkg{appraiseR}: an R Package for Real Estate Appraisals}

\Plainauthor{Luiz Fernando Palin Droubi, Willian Zonato, Norberto Hoccheim}
\Plaintitle{\pkg{appraiseR}: an R Package for Real Estate Appraisals}
\Shorttitle{\pkg{appraiseR}: an R Package for Real Estate Appraisals}

\Abstract{
The motivation behind \pkg{appraiseR} package comes from the need for
better Real Estate Appraisals (REA) in Brazil, but not only. Brazilian
standardization in this area is relative modern when compared to other
countries, since it determines the use of inference methods whenever its
possible to do so. On the other hand, the software industry in Brazil is
not well advanced in this area. There are several different commercial
softwares for REA, but none of them is able to do basic tests like
Breusch-Pagan test for heteroskedasticity. Another problems comes from
the softwares interfaces and the way people learn to use them, without
any good statistical strategy. We pretend that with a free software
package (and good interfaces built with shiny) we can improve teaching
people to do better statistics.
}

\Keywords{real estate, appraisals, \proglang{R}, shiny}
\Plainkeywords{real estate, appraisals, R, shiny}

%% publication information
%% \Volume{50}
%% \Issue{9}
%% \Month{June}
%% \Year{2012}
%% \Submitdate{}
%% \Acceptdate{2012-06-04}

\Address{
    Luiz Fernando Palin Droubi\\
  SPU\\
  First line Second line\\
  E-mail: \email{luiz.droubi@planejamento.gov.br}\\
  URL: \url{http://droubi.me}\\~\\
      Willian Zonato\\
  SPU\\
  First line Second line\\
  E-mail: \email{willian.zonato@planejamento.gov.br}\\
  
      Norberto Hoccheim\\
  UFSC\\
  First line Second line\\
  E-mail: \email{hochheim@gmail.com}\\
  
  }

\usepackage{amsmath}

\begin{document}

\section{The appraiseR package}\label{the-appraiser-package}

There is almost nothing in \pkg{appraiseR} built specially to Real
Estate Appraisals. Code used to do inference for REA can be also be used
in any other areas. Specially the \proglang{shinyapps}.

\section{Statistical Strategy}\label{statistical-strategy}

Faraway \citeyearpar[p.57]{faraway2004linear} has a good approach to
what can be a good statistical strategy. With simple tools widely
available in R for exploratory analysis, diagnostic, transformation and
variable selection one can almost ever find a good model. But, according
to Faraway, its easy even to a well trained analyst to be confused when
and where to use it, which can lead to not so good models and maybe
worse predictions.

According to Faraway, there is no proper tool utilization order. An
analysis may begin by the variable transformations, other from residual
analysis of a first linear model, and so on. Another way of starting an
statistical analysis is with forward or backward variable selection. But
Faraway recommends that the following order to be used:

Diagnostic→Transformation→Variable Selection→New diagnostic

It is not clear what Faraway mean by diagnostic, but we think is very
important to emphasize that good exploratory can lead to good models.
Faraway also says that is better to have in mind what objective has the
statistical analysis: in REA we are always interested in good
predictions.

\section{The reality of REA}\label{the-reality-of-rea}

Problems with reliable data are always present when dealing with REA.
Official transactions data are not always reliable, specially in Brazil
since tax evasion is a constant, so the transactions prices are biased.

Advertising prices are fine, but it is clearly not the real value of the
building, since there is always negotiation that leads to discounts.

We also think that dealing only with offers prices may also leads to
Real Estate bubbles. In Brazil it did not happen any Minsky moment in
Real Estate assets or something similar to USA. We think this is due to
the fact that here there was not that sub-prime loans problem, but there
is almost a consensus that there was a bubble but here and it is
disinflating slowly, due to nominal rigidity prices.

\section{Brazilian commercial softwares for
REA}\label{brazilian-commercial-softwares-for-rea}

Brazilian commercial software have any power if compared to
\proglang{R}. And teaching with commercial software is not the ideal
procedure, first because we think code must always be available to
critics, and teaching people with a commercial software is also not a
good idea, since they might be interested in using another software.

Our initial strategy for solving this problem was to build an R package
that could easily lead us to develop \proglang{shinyapps} with it. We
consider that shiny reactive programming is very useful for teaching,
and it is also difficult to convince people who are not familiar with
programming to start using \proglang{R}, but shiny can be very
attractive.

\href{https://droubi.shinyapps.io/REApp/}{REApp} was built with the
intention to allow people who are already users of other commercial
software to migrate to the use o \proglang{R}.

Briefly, it allows users to upload excel tables and uses the
\code{bestfit} \pkg{appraiseR} function to build tables of several
different models with different transformations for each variable in the
model. In the sidebar user can control which variables will be used in
the model, which transformations should be used to build the table,
which model to test or use for predictions and the model outliers. Every
chosen option affects reactive functions that updates the models
automatically. Tests available for normality and homoscedasticity were
chosen based in the standard, which includes Kolgomorov-Smirnov,
Breusch-Pagan, and others. The majority of the tests comes from other
packages, like \pkg{lmtest}, \pkg{normtest} and \pkg{nortest}, which is
far beyond the tests available in the commercial software.

Special attention have been taken with exploratory analysis, which is a
thing \proglang{R} can do better than any other software.

Functions to allow people to use box plots, histograms and scatter plots
were made available by \pkg{REApp}, coming not only from \pkg{appraiseR}
package but also from \pkg{mosaic} and others.

\section{Benchmarking}\label{benchmarking}

\section{Datasets available}\label{datasets-available}

There are a dozen of data sets extracted from previous teaching
material, like \code{jungle}, \code{trindade} and others. Almost all of
them are available with a specific purpose, i.e., for teach something
different, like Spatial Regression techniques. Some of them are
\code{SpatialPointsDataFrames}, like \code{centro_2015},
\code{centro_2013_2015}, \code{itacorubi_2015} and
\code{trivelloni_2005}.



\end{document}

